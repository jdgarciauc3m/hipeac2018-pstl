\begin{frame}[t]{Non-numeric algorithms}
\begin{itemize}
  \item Do not require value that value type is a numeric type.
    \begin{itemize}
      \item Mor general than numeric types.
    \end{itemize}
  \item Examples:
    \begin{itemize}
      \item \cppid{for\_each()}.
      \item \cppid{for\_each\_n()}.
    \end{itemize}
\end{itemize}
\end{frame}

\begin{frame}[t,fragile]{For Each}
\begin{lstlisting}[]
template <class ExecutionPolicy, class InputIterator, 
          class Function>
void for_each(ExecutionPolicy && ep, 
              InputIterator first, InputIterator last, Function f);
\end{lstlisting}
\begin{itemize}
  \item Applies \cppid{f} to every value in range \cppid{[first,last)}.
\end{itemize}
\begin{block}{Example}
\begin{lstlisting}[basicstyle=\scriptsize]
long count_primes(const vector<int> & v) {
  std::atomic<long> count;
  for_each(par, begin(v), end(v), [&](int x) {
    if (is_prime(x)) { count++; }
  });
  return count;
}
\end{lstlisting}
\end{block}
\end{frame}

\begin{frame}[t,fragile]{For Each}
\begin{lstlisting}[]
template <class ExecutionPolicy, class InputIterator, 
          class Size, class Function>
void for_each_n(ExecutionPolicy && ep, 
                InputIterator first, Size n, Function f);
\end{lstlisting}
\begin{itemize}
  \item Applies \cppid{f} to every value in range \cppid{[first,first+n)}.
\end{itemize}
\begin{block}{Example}
\begin{lstlisting}[basicstyle=\scriptsize]
long count_primes(const vector<int> & v, int k) {
  std::atomic<long> count;
  for_each_(par, begin(v), k, [&](int x) {
    if (is_prime(x)) { count++; }
  });
  return count;
}
\end{lstlisting}
\end{block}
\end{frame}

\begin{frame}[t,fragile]{But...}
\begin{itemize}
  \item \cppid{for\_each} is not usually the best way to do things.
\end{itemize}
\begin{block}{Example}
\begin{lstlisting}[basicstyle=\scriptsize]
long count_primes(const vector<int> & v) {
  return count_if(par, begin(v), end(v), [&](int x) {
    return is_prime(x);
  });
}
\end{lstlisting}
\end{block}
\end{frame}
