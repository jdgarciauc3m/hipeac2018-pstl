\begin{frame}[t]{Numeric algorithm}
\begin{itemize}
  \item Algorithms for \textgood{numeric sequences}.
    \begin{itemize}
      \item \textmark{Reduction}:
        \begin{itemize}
          \item \cppid{reduce}.
          \item \cppid{transform\_reduce}.
        \end{itemize}
      \item \textmark{Scan}:
        \begin{itemize}
          \item \cppid{exclusive\_scan}.
          \item \cppid{inclusive\_scan}.
          \item \cppid{transform\_exclusive\_scan}.
          \item \cppid{transform\_inclusive\_scan}.
        \end{itemize}
    \end{itemize}
  \vfill
  \item Also sequential versions without execution policy.
\end{itemize}
\end{frame}

\begin{frame}[t]{GENERALIZED\_SUM}
\begin{itemize}
  \item Building block for defining reductions.
  \vfill
  \item \cppid{GENERALIZED\_SUM(op, a$_1$, \ldots, a$_n$)}:
    \begin{itemize}
      \item \cppid{op(GENERALIZED\_SUM(op,b$_1$,\ldots, b$_k$), GENERALIZED\_SUM(op,b$_l$, \ldots, b$_n$))}.
      \item \cppid{b$_1$, \ldots, b$_k$,b$_l$, \ldots, b$_n$} is a permutation of \cppid{a$_1$, \ldots, a$_n$}.
    \end{itemize}
  \vfill
  \item \textmark{Impact}:
    \begin{itemize}
      \item Subranges may be grouped and rearranged.
      \item Assumes \cppid{op} is commutative.
    \end{itemize}
\end{itemize}
\end{frame}

\begin{frame}[t,fragile]{reduce}
\begin{lstlisting}[]
template <class ExecutionPolicy, class ForwardIterator, class T, class BinaryOperation>
T reduce(ExecutionPolicy && policy, ForwardIterator first, ForwardIterator last, T init, BinaryOperation op);
\end{lstlisting}
\begin{itemize}
  \item \cppid{GENERALIZED\_SUM(op, init, *first, \ldots )}
  \item \cppid{op} Cannot \textbad{invalidate} the iterators.
  \item \cppid{op} Cannot \textbad{modify} values in the range.
  \item \cppid{reduce} might be \textbad{non-deterministic} when \cppid{op} es non associative or non-commutative.
\end{itemize}
\begin{block}{Example: Adding values in a vector}
\begin{lstlisting}[]
sum = reduce(par, begin(v), end(v), 0, [](int a, int b) { 
  return a+b; });
\end{lstlisting}
\end{block}
\end{frame}

\begin{frame}[t,fragile]{Defaulting operation in reduce}
\begin{lstlisting}[]
template <class ExecutionPolicy, class InputIterator, class T, class BinaryOperation>
T reduce(ExecutionPolicy && policy, InputIterator first, InputIterator last, T init);
\end{lstlisting}
\begin{itemize}
  \item Defaults to addition.
  \item \cppid{reduce(first,last,init,plus<>())}.
\end{itemize}
\vfill
\begin{block}{Example: Adding values in a vector}
\begin{lstlisting}[]
sum = reduce(par, begin(v), end(v), 0);
\end{lstlisting}
\end{block}
\end{frame}

\begin{frame}[t,fragile]{Defaulting operation and initial value in reduce}
\begin{lstlisting}[]
template <class ExecutionPolicy, class InputIterator>
T reduce(ExecutionPolicy && policy, InputIterator first, InputIterator last);
\end{lstlisting}
\begin{itemize}
  \item Defaults to addition.
  \item Defaults initial value to value type default value.
  \item \cppid{reduce(first, last, iterator\_traits<ForwardIt>::value\_type\{\}, plus<>())}.
\end{itemize}
\vfill
\begin{block}{Example: Adding values in a vector}
\begin{lstlisting}[]
sum = reduce(par, begin(v), end(v));
\end{lstlisting}
\end{block}
\end{frame}

\begin{frame}[t,fragile]{Transform reduce}
\begin{lstlisting}[]
template <class ExecutionPolicy, class InputIterator, class T, class UnaryOp, class BinaryOp>
T transform_reduce(ExecutionPolicy && policy, InputIterator first, InputIterator last, UnaryOp unary_op, T init, BinaryOp binary_op);
\end{lstlisting}
\begin{itemize}
  \item Performs a reduction using \cppid{binary\_op} 
        to the result of applying \cppid{unary\_op} to elements.
    \begin{itemize}
      \item \cppid{unary\_op} not applied to \cppid{init}.
    \end{itemize}
\end{itemize}
\begin{block}{Example}
\begin{lstlisting}[basicstyle=\scriptsize]
auto sum_square = transform_reduce(begin(v), end(v), 
  [](double x) { return x * x; },
  0,
  [](double x, double y) { return x + y; }
);
\end{lstlisting}
\end{block}
\end{frame}

\begin{frame}[t,fragile]{Word frequencies}
\begin{block}{Computing word frequencies}
\begin{lstlisting}
auto word_freq(const std::vector<std::string> & words) {
  using namespace std;
  using dictionary = map<string,int>;

  return transform_reduce(execution::par, begin(words), end(words),
    [](string w) -> dictionary { return {w,1}; },
    dictionary{},
    [](dictionary & lhs, const dictionary & rhs) -> dictionary {
      for (auto & entry : rhs) { lhs[entry.first] += entry.second; }
      return lhs;
    }
  );
}
\end{lstlisting}
\end{block}
\end{frame}


\begin{frame}[t]{GENERALIZED\_NONCOMMUTATIVE\_SUM}
\begin{itemize}
  \item Building block for defining scan operations
  \vfill
  \item \cppid{GENERALIZED\_NONCOMMUTATIVE\_SUM(op, a$_1$, \ldots, a$_n$)}:
    \begin{itemize}
      \item \cppid{op(GENERALIZED\_NONCOMMUTATIVE\_SUM(op,a$_1$,\ldots, a$_k$), 
                      GENERALIZED\_NONCOMMUTATIVE\_SUM(op,a$_l$, \ldots, a$_n$))}.
    \end{itemize}
  \vfill
  \item \textmark{Impact}:
    \begin{itemize}
      \item Assumes \cppid{op} is not commutative.
      \item Still assumes \cppid{op} is associative.
    \end{itemize}
\end{itemize}
\end{frame}

\begin{frame}[t,fragile]{scan}
\begin{itemize}
  \item Two variants:
    \begin{itemize}
      \item \cppid{inclusive\_scan}: Performs \emph{scan} until \emph{i-th} element.
      \item \cppid{exclusive\_scan}: Performs \emph{scan} until just before \emph{i-th} element.
    \end{itemize}
  \item In both cases:
    \begin{itemize}
      \item Operations are performed in terms of \cppid{GENERALIZED\_NONCOMMUTATIVE\_SUM}.
      \item Operation can neither invalidate iterators nor modify elements.
    \end{itemize}
\end{itemize}
\begin{block}{Example}
\begin{lstlisting}[]
auto it_end = exclusive_scan(begin(v), end(v), begin(w));
\end{lstlisting}
\end{block}
\end{frame}

\begin{frame}[t,fragile]{exclusive scan}
\begin{lstlisting}[]
template <class InputIterator, class OutputIterator, class T, 
          class BinaryOp>
OutputIterator exclusive_scan(InputItertor first, InputIterator last, OutputIterator out, T init, BinaryOp op);
\end{lstlisting}
\begin{itemize}
  \item For every \cppid{i} in range \cppid{[out, out + (last-first))}:
    \begin{itemize}
      \item Assigns to \cppid{*i} the value
        \cppid{GENERALIZED\_NONCOMMUTATIVE\_SUM(op, init, *first, \ldots, *(first+(i-out)-1))}.
    \end{itemize}
  \item Returns iterator to the end of resulting range.
  \vfill
  \item Omitting \cppid{op} is equivalent to:
    \begin{itemize}
      \item \cppid{exclusive\_scan(first, last, out, init, plus<>())}
    \end{itemize}
  \item \cppid{init} cannot be omitted.
\end{itemize}
\end{frame}

\begin{frame}[t,fragile]{inclusive scan}
\begin{lstlisting}[]
template <class InputIterator, class OutputIterator, class T, 
          class BinaryOp>
OutputIterator inclusive_scan(InputItertor first, InputIterator last, OutputIterator out, T init, BinaryOp op);
\end{lstlisting}
\begin{itemize}
  \item For every \cppid{i} in range \cppid{[out, out + (last-first))}:
    \begin{itemize}
      \item Assigns to \cppid{*i} the value 
        \cppid{GENERALIZED\_NONCOMMUTATIVE\_SUM(op, init, *first, \ldots, *(first+(i-out)))}.
    \end{itemize}
  \vfill
  \item Omitting \cppid{op} is equivalent to:
    \begin{itemize}
      \item \cppid{exclusive\_scan(first, last, out, init, plus<>())}
    \end{itemize}
  \item If \cppid{init} is omitted, it is not included in the sum.
\end{itemize}
\end{frame}

\begin{frame}[t,fragile]{Example}
\begin{block}{Computing CDF from histogram}
\begin{lstlisting}[]
vector<int> histogram = compute_histogram();
vector<int> cdf(histogram.size(), 0);
inclusive_scan(par, begin(histogram), end(histogram),
  begin(cdf));
\end{lstlisting}
\end{block}
\end{frame}

\begin{frame}[t,fragile]{transform + exclusive\_scan}
\begin{itemize}
  \item One version for \cppid{transform\_exclusive\_scan}
\end{itemize}
\begin{lstlisting}[]
template <class InputIterator, class OutputIterator, 
          class UnaryOp, class T, class BinaryOp>
OutputIterator transform_exclusive_scan(
  InputItertor first, InputIterator last, OutputIterator out, 
  UnaryOp unary_op, T init, BinaryOp binary_op);
\end{lstlisting}
\end{frame}

\begin{frame}[t,fragile]{transform + inclusive\_scan}
\begin{itemize}
  \item Two versions for \cppid{transform\_inclusive\_scan}
\end{itemize}
\begin{lstlisting}[]
template <class InputIterator, class OutputIterator, 
          class UnaryOp, class BinaryOp>
OutputIterator transform_inclusive_scan(
  InputItertor first, InputIterator last, OutputIterator out, 
  UnaryOp unary_op, BinaryOp binary_op);


template <class InputIterator, class OutputIterator, 
          class UnaryOp, class BinaryOp, class T>
OutputIterator transform_exclusive_scan(
  InputItertor first, InputIterator last, OutputIterator out, 
  UnaryOp unary_op, BinaryOp binary_op, T init);
\end{lstlisting}
\end{frame}

\begin{frame}[t,fragile]{Example}
\begin{block}{Computing CDF from an histogram}
\begin{lstlisting}[]
vector<int> histogram = compute_histogram();
vector<int> cdf(histogram.size(), 0);
transform_inclusive_scan(par, begin(histogram), end(histogram),
  begin(cdf)
  [](int x) {
    if (x<0) return 0;
    if (x>255) return 255;
    return x;
  }
  [](int x, int y) { return x+y; });
\end{lstlisting}
\end{block}
\end{frame}
